% tip to kill the wrong type
%!TEX program = xelatex
\documentclass[10pt,journal,compsoc]{IEEEtran}
\IEEEoverridecommandlockouts
% The preceding line is only needed to identify funding in the first footnote. If that is unneeded, please comment it out.
\usepackage{cite}
\usepackage{amsmath,amssymb,amsfonts}
\usepackage{algorithmic}
\usepackage{graphicx}
\usepackage{textcomp}
\usepackage{xcolor}
%\usepackage{}
\usepackage{algorithm}
\usepackage{xeCJK}
\def\BibTeX{{\rm B\kern-.05em{\sc i\kern-.025em b}\kern-.08em
    T\kern-.1667em\lower.7ex\hbox{E}\kern-.125emX}}
\usepackage{fancyhdr}
%\pagenumbering{roman}

%%
%kErrb3rt:玩家的游玩路线是这样的,从主世界开始,首先发展自己(在"apocolypse"的监视之下,并且由于出生之日apocolypse就把监管植入了玩家,所以玩家在哪个维度都受到信用点管控(我等下补充,你不用发散这个部分),然后玩家随着史诗,找到了死亡之冠,与溺尸王大战,最终打败溺尸王,开启地狱的进入权限(此时玩家击碎沉星导致的全世界地狱门开启,以及拿到了太平洋之风),到达地狱后,前往Tyrant暴君的宫殿,暴君依然存活,只是不爱搭理生命,但是他感受到了那股风,于是玩家和暴君大战,玩家击败Tyran后,得到了暴君之镐(一把武器)和「echo of vOid」虚空回响,一个罗盘(损坏的罗盘),当玩家回到主世界,立刻被apocolypse本体留下的监听部分发现,并且夺取了虚空回响,但是天启的残存部分没有料到,天启本体留下的这个监听其实是针对残存部分的,这个监听得到了力量后,自毁开启了末地传送门,但是只有「生命思想体」才能进入,天启的分身并没有思想,这部分对于玩家来说,只是打开了末地开启权限的作用,但是对于剧情还是有点重要的。
%玩家此时直接死亡,然后时间回到最初,玩家再次诞生,从新走过之前的路,不同的是溺尸王改为了soul of 桑恩,暴君改为了卢纳,显然这是Nova在阻止玩家见到他,因为第一轮回的时候Nova就混在玩家之中
\begin{document}

\section{详细版本}
\subsection{众神的历史}
\subsubsection{关键事件--桑恩和卢纳的冲突篇}
上古时期,世界存在着三种势力,盘踞在主世界的雷暴之神桑恩,在下届世界的生命之神卢纳,在终末之地的虚无之神卡冈图雅,原本完美的世界可以一直走下去直到....生命之神腐败了,卢纳为了和践行生命的尊严,也是卢纳对下届生命自由的承诺,私自在主世界播撒下届的生命之种,直接使主世界的土壤於下届接壤,但是令卢纳没想到的是,下届的生命接触到主世界的土壤迅速腐化,下届世界丧失了生机,为了弥补对下届世界造成的后果卢纳私自离开神位,他庞大的身躯只能进入位于世界之眼的世界通道,当桑恩见到他的老朋友的时候,卢纳已经凋零为枯骨,但是依旧踏进了主世界的土壤,一瞬间,泥土石头开始燃烧腐化不可理解事物开始扩散.....

为了阻止卢纳进入主世界,桑恩使用其锻造之神器「太平洋之风」三叉戟,与卢纳交战,可是卢纳强大的意志似乎直接决定他是不可战胜的,那种强大的意志,又或是一种执念,直接入侵到了桑恩的精神,只是他为察觉罢了。二者交战于海洋之上,海水沸腾引起的巨浪也盖不住卢纳降临于错误世界流出的熔岩,桑恩使用了雷霆的力量,击碎了卢纳的头骨,再击碎了自己的,一时间,二者坠入海洋。直到1日后,主世界不再有雷暴之神的身影却因为错误的人播下错误的种子,诞生出了第一个「自然的智慧造物」。下届从此分隔于主世界,两世界不再有往来。这一切都是源于桑恩得到卢纳本源的生命之力的一角后,把它锁死在了「沉星」之中,但是与其说是神器「沉星」,实际上是雷暴之神的头骨被其灵魂锻造出的容器。桑恩和卢纳被击碎的一部分力量一起,永远的消失在主世界,成为虚空之物。

卢纳当然没有毁灭,失去一部分力量的卢纳只能在下届尝试找到出去的办法,但是卢纳被关在下届这个牢笼里,卡冈图雅因为观察到了桑恩和卢纳的冲突,选择保全自己的一席之地而使用力量断绝了主世界和下届的所有联系。卢纳孤身一人,只感觉到空虚,他作为生命之神,现在所能创造的只不过是些凋零的生物罢了。卢纳那可怕的意志再次涌上脑海,他清楚的意识到是「太平洋之风」打败了他,他开始创造凋零生命,为的不是生命的意义,而是奴隶这些个体,在下届创造独属于卢纳本人的神器:「暴君」。当他拿到那把由扭曲的灵魂制造的十字镐的时候,他失去了作为神的最后一点尊严,从此卢纳被自己杀死了,下届变为地狱,而地狱里又多出一个「暴君」。
\subsubsection{卡冈图雅篇}
事实上,上古时期只存在两个神桑恩和卢纳,一切源于一个意外。主世界的雷暴之神桑恩并没有在他的世界露过面,主世界远古的住民「X」们,崇敬着他们的神。某一天,一个错误「入侵者」陨石坠落于主世界,其包裹着弱小的种族「天顶(Zenithes)」的一员卡冈图雅,这个种族本身处于无尽的轮回之中,导致其科技无法发展。对于「天顶」期望卡冈图雅坐着「入侵者」入侵这个世界的期望不同,「X」认为他就是他们的神,并且引导着其他的族人一起供奉他,因为卡冈图雅的意识已将这个世界的整个时间线阅览不知多少遍了,他清楚的直到接下来会发生什么。「X」与外族接触的事情被雷暴之神知晓,雷暴之神没有愤怒,而是选择去和卡冈图雅接触,届时,如同白日一般的闪电降临在大地上,「X」真正的神祇,桑恩和他们见面了,当桑恩和卡冈图雅接触的一瞬间,似乎是一种预言的感受划过,桑恩看到了凋零的卢纳崩坏的主世界,然后才意识到从刚才开始他就在自己的潜意识里知道自己会「先踏出左脚,还是先踏出右脚」,桑恩被不属于这个世界的不可理喻之物震撼,把意识剥离暂时离开主世界前往了下届,而卡冈图雅则是空洞的看着桑恩的躯体,“你会选择创造一个新世界的,相信我,没有出过错....”。随后由于卡冈图雅的影响,整个主世界,都陷入了时间的停滞,这里不断重复着歌声,雷暴,唯一在的只有卡冈图雅,以及他的信众们「X」从卡冈图雅那里获得到了天顶一族的诅咒,被困在每一个时间点。

桑恩不相信自己创造的世界就此陨灭,从下届冷静后返回,但是他所看到的只有白和黑两种颜色,白色的石头,黑色的身影,他似乎无法注视某一个身影过久,因为他们会在他眨眼的瞬间消失,“你看到的只是时间的一角罢了,「X」们的时间线也完成了闭合,和天顶的所有人一样....”,桑恩痛苦的注视着他所创造的世界化为虚空,卢纳注意到了桑恩的波动,也降临到主世界,同样被这超越理解的一幕震撼。卡冈图雅出现在二者的面前低声说:“我的信使们存在于每一个维度,有时间的地方就有我的信使,你们被虚空拒绝了,所有请回吧”然后低沉的面容挥动着手,抹除了主世界的全部存在痕迹,二位神明也失去了记忆,对他来说这里就是时间线结束的地方「The End」,而信徒们会永远陪伴在卡冈图雅左右。

卡冈图雅一边回想着时间,他知道母星上那些被「选中」的人,已经死了,他知道天顶只剩下他一个了。不过他们的时间会因为宇宙而再重来的,他这样想,天顶的任何人都无法想象大灾难前活下来的是这个拒绝对另一个世界发动侵略的遗孤,因为其他人的时间都在此之前归零了。尽管不想发动侵略,但是带有诅咒的生物,只需要一个就可以决定他所踏入的世界的命运,他从一开始就知道主世界将变成「末地」但是没有任何办法阻止。
\subsubsection{桑恩篇}
桑恩的记忆只能追溯到,他创造的世界尽管无法自然诞生出智慧的生灵,但是却具有无穷的宝藏。和卢纳的生命繁荣的世界不同,桑恩所占有的只是一片远古的大陆,充斥着植物生命,但是却没有足够智慧的动物构建出文明。桑恩与卢纳交好,他似乎感觉他和卢纳的友谊不仅仅开始于时间的起点,或者比宇宙本身还要古老。「末地的原住民」桑恩是这样称呼这群漆黑的信使的,主世界庞大的地界有数不清的末地人来来往往,但是似乎没有在发展任何东西,反过来桑恩有一瞬间觉得那些末地人对他投过来的目光是如此的怜悯,似乎在乞求他的原谅,不可理解之事仍在发生着....

远古时期,主世界覆盖着众多的植物生命以及小型的动物,令桑恩和生命之神卢纳不理解的是,为什么不论如何使用主世界的生命之种播撒,这个世界都无法补上最后一角,令桑恩绝望的是作为主世界的至高神明却没有统治半点东西,有时也对自己总想管点什么的思想嗤之以鼻,但是又害怕那些才是真心话。桑恩平静的日子由来自末地的信使们打破,这些漆黑的教徒散发着紫色的幽光,令桑恩时时觉得毛骨悚然。信使把桑恩带到末地,随着巨大的龙吼声,桑恩见到了卡冈图雅,以及一条黑色的,不详的龙。“我知道你在进行虚空的探索,但是,你创造的生命实在是...”桑恩这样评价这条黑色的无神的龙。卡冈图雅在他们的谈话结束之时,赠送给了桑恩一袋蓝色的种子,”只要把这个播种在主世界,就一定会改变你那边的情况的“,桑恩根据末地人的提示,把种子播种在一个地下的区域,但是并没有什么改变,这些只是另一种植物的种子罢了,那些冒着幽幽蓝光的地衣覆盖了土地,倒不如说感染了他,似乎这种植物对于动物的死亡有极强的反应,只要周围有动物死去,这种植物以短时间可见的速度覆盖到尸体上,但是尸体却不会腐坏。

桑恩所未料到的,一切只是卡冈图雅预言的未来的轨道,时间的列车从未出轨过,包括这一次。桑恩的利器「太平洋之风」由主世界配对的元素制造而成,除了桑恩,永远也没有人知道那是一把什么样的武器,只是更具桑恩的化身进行相应的变化。「太平洋之风」激起的最后雷电在海上发生了,卢纳喷涌着岩浆和残缺不全的桑恩一起落入了大海,一瞬间高温导致海上云雾缭绕,桑恩的意识在失去的最后一刻看到了,那些冒着幽幽蓝光的植物包裹他的身体,深深的扎根于地下,他第一次看见了虚空。虚空是如此的安静,仿佛有无数的猎手以声音为食。

只有卡冈图雅知道,他没有接见过什么桑恩,桑恩来的末地只是那群末地人搞的鬼,他们研究虚空的力量导致的灾难不能从末地蔓延出去,实验性的种子结合主世界的土壤可以得到战争利器「warden(坚守者)」,他们想把虚空的回声灾难带到主世界,还在主世界地下建立巨大的连通虚空的传送门。但是GARGANTAU能做什么呢,一切都写好了,卡冈图雅在桑恩被回声吞噬之前,把桑恩困在时间的轮回中了,他死亡的那一秒暂时不会到来,但是回声污染的桑恩孤独的灵魂却要和桑恩的躯体一起轮回了。

\subsubsection{卢纳篇}
主机迭代96报告:旧神史-卢纳篇

在故事的开始,下届和主世界是从一个巨大的岩浆状态分离出来的,自然的冷却下,二者如同孪生兄弟一样,几乎没有差别,两位神桑恩和卢纳,因为天空只有一片,所以本来完全平衡的世界被打破,二位神本来就是很好的朋友,所以卢纳就把天空让给了桑恩的世界,自己的只有巨大黑暗的穹顶,但是他也很满足,肆意播撒着生命的种子,很快,主世界诞生了文明「X」,下届也诞生出文明「先驱者」,先驱者们在下届进行文明的发展,建立了城堡和堡垒,下届的海洋上还有奇异灵魂生物,被先驱者作为交通工具使用。

随着「第三个世界」的到来,卢纳自然不知晓本来世界的样貌,只是继续他的工作。但是很快,先驱者们发现,他们的极限仅仅是到达那个漆黑的穹顶,于是他们向卢纳祈祷想要无限的广阔的天地,卢纳作为他们的神,自然没有忽视他们的需求,于是许下承诺,必定带给他们无限的世界,于是卢XXXXXXXX 0X008120主机数据错误...........;;;;;;''''''

「天启」报告:已获得主机权限,后门已注入,恢复主机工作.  XXXXXXXXXXXX   0x2301..........

主机迭代97报告:旧神史-卢纳篇

,于是卢纳从主世界采取土壤,在自己的住所里进行了测试,发现主世界的土壤可以有效让穹顶变得松动和脆弱,未知的只剩下穹顶究竟有多厚.卢纳发动各个生物,努力开凿穹顶,利用稀少的主世界土壤,他们竟然遭开了巨大的空间,虽然还是没有看到穷顶之外的世界,但是随着进度的推进所有下届的生灵都充满了斗志.直到--卢纳意识到,主世界是被选中的世界,所谓凿开的空间,不过是把穹顶用主世界的土壤转化为了天空,他们需要的不是努力,而是一开始的祝福.于是卢纳进入主世界,然后被愤怒的桑恩以生命的代价夺取了神格,打退回了下届.当卢纳再次起身观察四周时,他残破的灵魂能看清的只是漆黑的,充满熔岩的高温地狱,以及远古著名所化为的凋零之物。卢纳残缺的灵魂只剩下追求自由的偏执,偏执由化为悲愤和暴力,他开始奴隶那群没有生命的凋零物体,以及那些游荡的亡灵,不断的开凿穹顶,同时又下令从天穹的材料以及自己生命之神再也托举不起的生命核心为材料,锻造了「暴君」十字镐,作为下届君王的象征。他知道,他自己再也没有机会踏足主世界,但那些亡灵「X」们,末地人们,在他的耳边低语,他知道了一个震撼的实事:主世界诞生出了文明,同时他也直到了,溺尸王的幽魂还游荡在主世界的耳语。于是,卢纳再次从生命的合金里汲取力量,锻造了「死亡之冠」并且吩咐「X」的一员去戴给桑恩的幽魂--溺尸王的头上,一瞬间,溺尸王被夺取力量碎裂,但是这蕴含着全部主世界力量的王冠又怎么可能穿越地狱门,卢纳只好通过控制在下届游荡的「X」们往返于主世界和下届,散布着那个--死亡之冠囚禁的溺尸王拥有的强大神器「太平洋之风」的传言。而卢纳自己则等待着,文明的力量得到那把神器解开自己的枷锁.....





\end{document}
